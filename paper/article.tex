\documentclass[11pt,a4paper]{article}

% Core packages
\usepackage[utf8]{inputenc}
\usepackage[T1]{fontenc}
\usepackage{lmodern}
\usepackage{microtype}

% Math and symbols
\usepackage{amsmath,amssymb,amsfonts}
\usepackage{mathtools}

% Tables and figures
\usepackage{booktabs}
\usepackage{array}
\usepackage{multirow}
\usepackage{graphicx}

% Layout and formatting
\usepackage[margin=1in]{geometry}
\usepackage{setspace}
\usepackage{parskip}

% Code listings
\usepackage{listings}
\lstset{
  basicstyle=\ttfamily\small,
  breaklines=true,
  frame=single,
  xleftmargin=0.5em,
  xrightmargin=0.5em
}

% Hyperlinks (load last)
\usepackage[colorlinks=true,linkcolor=blue,citecolor=blue,urlcolor=blue]{hyperref}

% Bibliography
\usepackage{natbib}
\bibliographystyle{plainnat}

\title{Semantic Compression of LLM Instructions via Symbolic Metalanguages}
\author{Ernst van Gassen\\
  \textit{Independent Researcher}\\
  \texttt{evg.gassen@gmail.com}
}
\date{}

\begin{document}

\maketitle

\begin{abstract}
This paper introduces MetaGlyph, a symbolic instruction metalanguage designed to semantically compress natural-language prompts by encoding instruction meaning into compact symbolic forms. Unlike learned prompt compressors or constructed languages that require explicit decoding schemes, MetaGlyph leverages mathematical and logical operators with high pretraining exposure, testing whether such symbols function as intuitive instruction-semantic primitives. We evaluate MetaGlyph across four task families---selection and classification, structured extraction, constraint composition, and conditional transformation---comparing natural-language instructions, symbolic MetaGlyph instructions, and semantically incoherent control prompts. Results show that MetaGlyph achieves 62--81\% token reduction while preserving instruction semantics in a task- and operator-dependent manner. The membership operator ($\in$) shows the highest reliability across models (up to 33.3\%), while compositional operators like implication ($\Rightarrow$) and intersection ($\cap$) show zero fidelity without explicit specification. These findings establish semantic compression of the instruction channel as a viable design principle and identify productive directions for developing robust symbolic instruction languages.
\end{abstract}

\medskip
\noindent\textbf{Keywords:} prompt compression, symbolic metalanguages, instruction semantics, large language models, prompt engineering

\section{Introduction}

Large language models are typically controlled through natural-language prompts. In practice, a prompt functions as an instruction language: it specifies a task, defines constraints, and often encodes a procedure, all in prose.

This works well enough to be widely adopted, yet it has two recurring weaknesses. First, prompts are verbose relative to the amount of intent they express, especially when constraints must be made explicit and unambiguous. Second, prompt behavior can be fragile: small changes in wording, ordering, or punctuation can produce disproportionately different outputs.

These weaknesses are not merely stylistic. They create operational costs through increased token usage, and they create epistemic costs through ambiguity and inconsistent interpretation. At the same time, modern models are pretrained on vast amounts of symbolic material, including mathematics, logic, programming languages, and technical notation, where meaning is frequently carried by compact glyphs and compositional structure.

This raises a basic question about instruction language: if models already internalize stable symbolic conventions from pretraining, can we communicate intent more directly through those conventions, using fewer tokens while preserving the same semantics? The practical significance is immediate. Token usage is a primary driver of inference cost and often correlates with latency, which motivates a growing body of work on prompt compression. Most of this work aims to reduce contextual content, for example by pruning retrieved passages, summarizing long inputs, or deleting ``unimportant'' words while trying to preserve downstream task performance.

A related line of work can be described as prompt distillation or prompt optimization, where shorter prompts are learned from longer prompts, frequently in a task-specific manner or via soft prompting techniques. These approaches target an important bottleneck, but they treat the instruction language itself as largely fixed. In many real prompts, the instruction portion is not a marginal overhead---it is a substantial fraction of the total length, particularly when the user must specify scope, exclusions, and output format.

Moreover, natural language is intrinsically underspecified as a formal control channel. Words like ``only,'' ``except,'' ``roughly,'' and ``summarize'' act like operators, yet their scope and interaction are often unclear in prose, especially when multiple constraints are stacked. If instruction semantics can be expressed more compactly and more structurally, then one can plausibly gain both efficiency and reliability---not by compressing the content being processed, but by compressing the language used to specify what should be done with that content.

The research gap is therefore not ``can prompts be shortened,'' but ``can the instruction language be compressed while preserving semantics, without learning a new language through a system prompt or model tuning.''

Existing constructed prompt languages, such as SynthLang, show that symbolic command languages can be executed by large language models when an explicit decoding scheme is provided. SynthLang is a symbolic prompt language described in an open-source repository \citep{synthlang2024} and depends on a standardized system prompt to specify its semantics. That setup is useful, but it conflates two sources of performance: the metalanguage itself, and the externally provided semantics that teach the model how to interpret it.

Other research explores symbol tuning with arbitrary labels, showing that models can learn to associate tokens with behaviors through training or alignment, even when the symbols themselves carry no inherent semantic meaning \citep{wei2023symbol}. Prompt compression systems such as LLMLingua, and related context pruning methods, achieve efficiency gains by manipulating or removing contextual content rather than by redesigning the instruction language itself \citep{jiang2023llmlingua,jiang2024llmlingua2}.

A related line of research uses symbolic and compressed prompt forms in adversarial settings to probe or bypass model safety mechanisms. For example, \citet{bethany2024jailbreaking} demonstrate how encoding harmful instructions as symbolic mathematical representations can induce unsafe model outputs, highlighting that compact symbolic transformations can materially affect aligned behavior even though such work is not constructed as controlled instruction semantics.

In contrast, our central question concerns intuitive, pretrained semantics: do widely used glyphs such as implication, inclusion, conjunction, and negation function as instruction-semantic primitives that a model can interpret without additional teaching? This question sits naturally at the intersection of linguistics and formal language design. It also resonates with older traditions outside machine learning.

In legal drafting, Allen's work on symbolic logic argues that compact symbolic forms can reduce ambiguity and increase precision when specifying obligations, conditions, and logical dependencies \citep{allen1957symbolic}. The parallel is direct: a prompt is a specification of intent, and if intent can be represented as structured constraints rather than as prose, then the instruction language becomes a deliberately designed metalanguage rather than an ad hoc paragraph of text.

I address this gap by introducing MetaGlyph, a symbolic instruction metalanguage designed to semantically compress natural-language prompts by encoding instruction meaning into compact symbolic forms. MetaGlyph is not a learned compressor and not a private codebook enforced by a system prompt. Instead, it is a constrained design that intentionally reuses glyphs with high pretraining exposure and relatively stable conventional meanings across domains.

The core idea is to treat certain symbols as grammatical devices for instruction specification---for example, mapping and transformation, membership and exclusion, conjunction and disjunction, and implication and equivalence. MetaGlyph therefore aims to increase semantic density per token, not by removing information from the prompt, but by expressing the same constraints with fewer, more structured tokens.

Our empirical goal is modest and falsifiable: we test whether MetaGlyph instructions produce behavior comparable to natural-language instructions under controlled conditions, and whether any observed gains persist when token count and superficial ``symbolic look'' are controlled. Concretely, we evaluate paired prompts across several task families that stress different aspects of instruction semantics, including selection and classification, structured extraction, constraint composition, and simple conditional reasoning.

For each task, we compare a natural-language instruction, a MetaGlyph instruction, and a shape-matched symbolic control that preserves visual and token-level characteristics while breaking the intended semantics. We report semantic equivalence metrics and error types, and we treat token reduction as a descriptive property rather than as a proxy for runtime improvements that would require system-level instrumentation.

The remainder of the paper proceeds as follows:
\begin{itemize}
    \item Section~\ref{sec:framing} motivates the metalanguage perspective and distinguishes semantic compression of instructions from context compression and learned prompt distillation, while situating the approach within both contemporary prompting research and earlier work on symbolic specification.
    \item Section~\ref{sec:related} presents related work spanning prompt compression, symbol tuning, and structured prompting.
    \item Section~\ref{sec:methodology} defines MetaGlyph, including its design principles, its minimal glyph inventory, and an explicit grammar for composing constraints and transformations.
    \item Section~\ref{sec:results} reports results across tasks and models, with ablations by symbol type and analyses of failure modes.
    \item Section~\ref{sec:analysis} provides a linguistic analysis of the observed behaviors, focusing on syntactic scope, compositionality, and pragmatic stability across contexts.
    \item Section~\ref{sec:discussion} discusses implications for prompt design, interpretability, and efficiency, and clarifies what the experiments do and do not establish about system-level performance.
    \item Section~\ref{sec:conclusion} concludes with limitations, practical guidance for reuse, and directions for extending symbolic instruction metalanguages beyond single-turn prompting.
\end{itemize}

\section{Conceptual Framing: Instruction Languages as Symbolic Metalanguages}
\label{sec:framing}

\subsection{Prompts as Metalanguages Rather Than Messages}

In common usage, prompts are treated as messages addressed to a conversational agent. Functionally, however, prompts behave less like messages and more like specifications. They do not primarily convey information to be discussed, but rather define how information should be selected, transformed, constrained, or generated. From a linguistic perspective, this places prompts squarely in the category of metalanguages: languages used to describe operations over other representations.

Natural language is an expressive but inefficient metalanguage. It relies heavily on pragmatic inference, implicit scope, and contextual repair. When a prompt states ``summarize the text, include only technical points, exclude speculation, and format the result as JSON,'' it implicitly encodes multiple operations: transformation, filtering, negation, and formatting. These operations are linearized into prose, even though their logical structure is not linear. The model must reconstruct the intended structure from surface cues, while the human author must anticipate how that reconstruction will occur. This mutual inference process is fragile and verbose.

A practitioner-driven body of exploratory prompting practices treats prompt fragility as a diagnostic signal rather than as a defect. This approach, often referred to descriptively as promptology, emphasizes systematic probing of how models respond to minimal changes in phrasing, ordering, or symbolic form, with the aim of identifying regularities in model behavior. A recurring observation in such practice is that models often respond more reliably to compact, structured representations than to elaborated prose. From a metalanguage perspective, this suggests that large language models internalize certain operator-like semantics during pretraining, and that effective prompting may consist in activating those semantics rather than repeatedly restating them in natural language.

\subsection{Core Symbolic Operators as Semantic Primitives}

Mathematical and logical notation provides a repertoire of compact symbols that encode relational structure with high semantic density. Many of these symbols recur across domains and are therefore heavily represented in pretraining corpora. Their potential value as instruction-semantic primitives lies in three properties: stability of meaning, compositionality, and brevity.

Directional operators such as implication ($\to$, $\Rightarrow$) are especially salient. In informal prompts, transformation is typically expressed through verbs like ``convert,'' ``rewrite,'' or ``map,'' which vary in strength and scope. The symbol $\to$ consistently encodes directionality and outcome: input $\to$ output. In instruction contexts, it can naturally function as a transformation operator, signaling that one representation should be systematically transformed into another. The stronger implication symbol $\Rightarrow$ often appears in rule-like contexts, suggesting necessity or entailment rather than mere association. Models encounter both symbols in mathematical derivations, programming explanations, and formal specifications, reinforcing their differentiated roles.

Set-theoretic operators provide similarly compact encodings of selection and constraint. Membership ($\in$) expresses inclusion without ambiguity: $x \in S$ denotes that $x$ must belong to $S$. Its complement ($\notin$ or $\neg\in$) expresses exclusion just as directly. In natural language, inclusion and exclusion are often underdetermined by phrases such as ``include,'' ``only,'' or ``except,'' which rely on pragmatic repair. Symbolic membership makes the constraint explicit and local, reducing scope ambiguity.

Conjunction and disjunction are particularly important in instruction specification. Natural language ``and'' and ``or'' frequently obscure whether constraints are cumulative or alternative, inclusive or exclusive. Symbols such as $\cap$ and $\cup$, or their logical counterparts $\land$ and $\lor$, encode these distinctions explicitly. In an instruction metalanguage, $A \cap B$ unambiguously requires simultaneous satisfaction of two constraints, whereas $A \cup B$ allows either. This is especially valuable when prompts specify multiple criteria or categories, where misinterpretation of conjunction is a common source of error.

Negation ($\neg$) functions as a scope-sensitive operator that benefits greatly from symbolic representation. In prose, negation can be lexical (``not''), morphological (``non-''), or pragmatic (``avoid''), each interacting differently with emphasis and scope. Symbolic negation has a narrow and explicit function: it inverts a predicate. When used in instruction contexts, $\neg$ can clearly mark prohibited content or excluded categories without introducing additional syntactic complexity.

\subsection{Extended Operator Inventory and Edge Cases}

Beyond these core operators, a broader inventory of symbols offers additional expressive power for instruction metalanguages.

Quantifiers such as $\forall$ (universal) and $\exists$ (existential) encode scope over sets of elements. In natural language prompts, quantification is often implicit and fragile, expressed through words like ``all,'' ``any,'' or ``at least one.'' These terms can interact poorly with downstream constraints. Symbolic quantifiers make scope explicit. For example, $\forall x \in S$ can signal that a condition applies uniformly across all elements of a set, whereas $\exists x \in S$ signals that satisfying a condition for at least one element is sufficient. While quantifiers are less common in everyday prompting, they are deeply embedded in the model's exposure to formal reasoning tasks and therefore represent plausible instruction primitives for certain classes of tasks.

Subset relations ($\subseteq$, $\subset$) refine membership by encoding hierarchical constraints. Whereas $\in$ specifies element-level inclusion, $\subseteq$ specifies that one set must be entirely contained within another. In instruction contexts, this can express requirements such as ``all outputs must fall within this category,'' which are cumbersome to express precisely in prose.

Mapping and abstraction operators such as $\mapsto$ are frequently used in mathematics and programming to define functions without naming them. In an instruction metalanguage, $\mapsto$ can signal pointwise transformation or anonymized mapping, distinguishing it from global transformation signaled by $\to$. Composition ($\circ$) encodes sequential application of transformations, a concept that is often expressed awkwardly in natural language through phrases like ``first\ldots\ then\ldots''. Symbolic composition makes ordering explicit and compositional rather than narrative.

The vertical bar $|$ serves multiple roles across domains, including ``such that'' in set-builder notation, restriction in functions, and conditional probability. In instruction contexts, $|$ can act as a scope delimiter, clarifying which constraints apply to which entities. Its frequent appearance in formal definitions and programming syntax increases the likelihood that models treat it as a structural marker rather than as content.

These edge-case operators illustrate an important point: not all symbols are equally suitable as instruction primitives. Their usefulness depends on semantic stability, frequency in pretraining data, and the extent to which they already function as structural markers rather than content tokens. A symbolic metalanguage should therefore be selective rather than exhaustive, prioritizing operators whose semantics are reinforced across many contexts.

\subsection{Pretrained Semantics Versus Externally Imposed Decoding}

A central distinction in this work is between symbolic operators whose semantics arise from pretraining exposure and symbols whose meaning is imposed externally through an explicit decoding scheme. In constructed prompt languages such as SynthLang \citep{synthlang2024}, symbol interpretation is defined by a system-level specification, meaning that observed efficiency gains reflect both the structure of the metalanguage and the externally provided semantics. Our focus, by contrast, is on symbols whose operator-like behavior emerges without such specification.

The present framing instead asks whether some symbols already function as intuitive semantic operators for models. This is a stronger and more constrained claim. It does not assume that models reason formally, but that repeated exposure to structured symbolic usage has induced stable associations between certain glyphs and certain relational patterns. Promptology practice supports this hypothesis empirically, insofar as models respond consistently to particular symbols even when they are not explained.

Symbol tuning research shows that models can learn arbitrary symbol--behavior mappings, but those mappings lack intrinsic meaning and are therefore brittle outside the training context. In contrast, operators such as $\in$ or $\to$ carry semantics that are reinforced across domains, making them more likely to generalize as instruction primitives.

\subsection{Designing Instruction Metalanguages}

Viewing prompts as metalanguages reframes prompt design as a language-design problem. This idea has precedent in legal drafting, where \citet{allen1957symbolic} argued that symbolic logic can reduce ambiguity by making logical structure explicit. The same logic applies to prompting. A purely prose-based prompt is an ad hoc specification, whereas a prompt that incorporates symbolic operators begins to resemble a designed language, even if it remains informal and hybrid.

MetaGlyph adopts this perspective by selectively incorporating symbolic operators with high semantic stability and pretraining exposure. Its goal is not to replace natural language, but to compress it by offloading structural semantics onto symbols. The experiments that follow therefore test not whether symbolic prompts are shorter, but whether they preserve instruction semantics under controlled conditions. If they do, this establishes semantic compression of the instruction channel as a viable design principle, independent of any system-level optimization or learned decoding scheme.

\begin{table}[htbp]
\centering
\caption{Candidate Symbolic Operators for Instruction Metalanguages}
\label{tab:operators}
\begin{tabular}{@{}llll@{}}
\toprule
Operator & Domain Origin & Canonical Meaning & Instruction-Semantic Role \\
\midrule
$\to$ & Logic / Math & Directional implication & Transformation / mapping \\
$\Rightarrow$ & Logic & Strong implication & Rule-like constraint \\
$\in$ & Set theory & Membership & Inclusion / filtering \\
$\notin$ & Set theory & Non-membership & Exclusion \\
$\cap$ & Set theory & Intersection & Conjunctive constraint \\
$\cup$ & Set theory & Union & Disjunctive constraint \\
$\neg$ & Logic & Negation & Prohibition / exclusion \\
$\forall$ & Logic & Universal quantifier & Apply constraint to all \\
$\exists$ & Logic & Existential quantifier & At least one satisfies \\
$\subseteq$ & Set theory & Subset & Hierarchical constraint \\
$\mapsto$ & Math / CS & Anonymous mapping & Pointwise transformation \\
$\circ$ & Math & Composition & Sequential operations \\
$|$ & Math / CS & Such-that / restriction & Scope delimitation \\
\bottomrule
\end{tabular}
\end{table}

\section{Related Work}
\label{sec:related}

Work related to this paper spans prompt compression, prompt distillation, symbolic and structured prompting, and adversarial uses of symbolic encodings. While these strands overlap at a surface level, they differ fundamentally in how instruction semantics are treated: as fixed, learned, or intrinsic to the model.

\paragraph{Prompt Compression and Context Reduction.}
A substantial body of research addresses inference cost and latency by reducing prompt length through context compression. Methods such as LLMLingua and LLMLingua-2 prune, summarize, or selectively retain contextual tokens using auxiliary models or learned importance scores \citep{jiang2023llmlingua,jiang2024llmlingua2}. Related approaches similarly optimize retrieval pipelines and context window usage. These methods demonstrate that large token savings are possible without major performance degradation. However, they operate exclusively on the content supplied to the model. The instruction language itself is assumed to be fixed, and these approaches do not examine whether instruction semantics can be expressed more compactly or more structurally.

\subsection{Prompt Distillation and Learned Optimization}

Prompt distillation and optimization approaches aim to learn shorter prompts from longer ones, often through gradient-based tuning, reinforcement learning, or soft prompt representations. While such methods can produce compact prompts with strong task performance, the resulting representations are task-specific learned artifacts. Their semantics are defined by optimization objectives rather than by linguistic convention, which limits insight into whether compact forms are effective due to intrinsic semantic structure or learned correlations. In contrast, the present work does not learn new prompt representations, but asks whether pretrained models already interpret certain symbolic forms as instruction-semantic primitives.

\subsection{Symbol Tuning and Arbitrary Labels}

Symbol tuning research shows that language models can associate arbitrary tokens with behaviors when those associations are reinforced during training or alignment \citep{wei2023symbol}. These results demonstrate that symbol--behavior mappings need not be grounded in prior meaning. However, the symbols used in such settings are intentionally semantically empty prior to tuning. Their effectiveness depends on explicit learning signals rather than on conventions internalized during pretraining. This differs from our focus on mathematical and logical symbols whose meanings are reinforced across domains and therefore plausibly available without additional instruction.

\subsection{Constructed Prompt Languages and Explicit Decoding}

Constructed prompt languages such as SynthLang show that symbolic shorthand can be executed efficiently when interpretation rules are explicitly specified \citep{synthlang2024}. These systems rely on standardized system prompts that define how symbols should be decoded, enabling consistent behavior across models. While practically useful, such approaches conflate two factors: the expressivity of the symbolic metalanguage and the externally imposed semantics that teach the model how to interpret it. They therefore do not isolate whether symbolic operators function as intuitive instruction primitives for pretrained models.

\subsection{Structured Symbolic Prompting and Adversarial Encodings}

Several studies demonstrate that structured, non-prose prompts can improve reasoning performance. Pseudo-code prompting and chain-of-symbol prompting show that symbolic representations can outperform natural language in tasks involving planning or multi-step reasoning. These works establish that models can leverage symbolic structure, but they primarily evaluate task accuracy rather than semantic equivalence under controlled token budgets. Separately, adversarial research shows that symbolic mathematics encodings can bypass alignment constraints, underscoring that compact symbolic forms can materially affect model behavior \citep{bethany2024jailbreaking}. Such work is adversarial in nature and does not provide a constructive account of instruction semantics.

\subsection{Positioning of the Present Work}

In contrast to prior approaches, this work focuses on semantic compression of the instruction language itself, using symbolic operators whose meaning is plausibly grounded in pretraining exposure rather than learned or externally imposed. By separating instruction semantics from context compression, prompt distillation, and adversarial objectives, we identify symbolic metalanguages as a distinct and underexplored design space for LLM prompting.

\section{Methodology}
\label{sec:methodology}

\subsection{Overview}

Our goal is to test whether a symbolic instruction metalanguage can semantically compress instructions without relying on a system-level decoding scheme. We evaluate MetaGlyph by comparing:
\begin{enumerate}
    \item a natural-language instruction (NL),
    \item a MetaGlyph instruction (MG) that encodes the same constraints using symbolic operators, and
    \item a symbol-shaped control (CTRL) that matches the surface form of MetaGlyph while breaking its intended semantics.
\end{enumerate}

The core requirement is that any performance differences cannot be attributed to longer instructions, additional explanatory text, or uncontrolled tokenization artifacts.

\subsection{MetaGlyph Operator Inventory and Grammar}

MetaGlyph is operationalized as a minimal inventory of high-frequency mathematical and logical glyphs used as instruction-semantic primitives. We prioritize operators with stable cross-domain usage in mathematics, logic, and technical writing, and with strong exposure in pretraining corpora. The operator inventory used in the experiments is shown in Table~\ref{tab:metaglyph-ops}.

\begin{table}[htbp]
\centering
\caption{MetaGlyph Operator Inventory}
\label{tab:metaglyph-ops}
\begin{tabular}{@{}lll@{}}
\toprule
Operator & Function & Example Usage \\
\midrule
$\in$ & Membership / inclusion & \texttt{$\in$(mammal)} --- item belongs to category \\
$\neg$ & Negation / exclusion & \texttt{$\neg$(bird)} --- item must not be in category \\
$\cap$ & Intersection / conjunction & \texttt{$\in$(pet) $\cap$ $\in$(mammal)} --- both constraints apply \\
$\to$ & Transformation / mapping & \texttt{items $\to$ select} --- transform input to output \\
$\Rightarrow$ & Implication / conditional & \texttt{$\in$(admin) $\Rightarrow$ access=full} --- if-then rule \\
$\circ$ & Composition / sequencing & \texttt{filter $\circ$ sort} --- apply operations in order \\
\bottomrule
\end{tabular}
\end{table}

We implement a lightweight operator grammar with three fields:
\begin{enumerate}
    \item \textbf{Input anchor}: a variable naming the input (for example, \texttt{items}, \texttt{report}, \texttt{accounts}).
    \item \textbf{Constraint clause}: a compositional expression over predicates and sets (for example, \texttt{$\in$(mammal) $\cap$ $\in$(pet) $\cap$ $\neg$(bird)}).
    \item \textbf{Task clause}: an operation mapping the anchored input to an output representation (for example, \texttt{$\Rightarrow$ select $\circ$ sort(name)}).
\end{enumerate}

MetaGlyph remains hybrid: predicates and labels are expressed in ordinary tokens (for example, \texttt{mammal}, \texttt{technical}, \texttt{employee}), while scope and composition are expressed symbolically. MetaGlyph instructions never include legends, glossaries, or explanations of operator meaning.

\subsection{Prompt Variants}

For each test instance we construct three prompt variants that share the same input and differ only in the instruction segment.

\paragraph{Natural-language baseline (NL).} A verbose English instruction expressing the task and constraints in standard prose. NL instructions average 172 tokens across task families, with detailed explanations of each constraint, explicit scope markers, and format specifications.

\paragraph{MetaGlyph (MG).} A symbolic instruction expressing the same semantics using the operator grammar. MG instructions average 51 tokens across task families. MG never introduces a legend, glossary, or system prompt teaching semantics.

\paragraph{Symbol-shaped control (CTRL).} A prompt that preserves the symbolic structure and token footprint of MG but replaces coherent operators with semantically irrelevant glyphs. CTRL uses visually similar but meaningless symbols ($\odot$, $\otimes$, $\oplus$) in place of the operators ($\cap$, $\in$, $\Rightarrow$), while preserving bracket structure, layout, and token count.

All variants share a fixed wrapper format: an instruction block, followed by the task input. The instruction explicitly requests JSON output format.

\subsection{Token Control Protocol}

Tokenization is a primary confound in any claim about compression, especially when symbolic or multilingual characters are involved. We evaluate under a token-controlled protocol.

\paragraph{Token accounting.} We measure tokens using a simple whitespace-based tokenizer that provides consistent counts across model families. Token counts are computed separately for (a) the instruction segment and (b) the task input. The token-control constraint is enforced only on the instruction segment, since the input is held constant across variants.

\paragraph{Budget matching.} For each task family we create fixed NL, MG, and CTRL instruction templates. CTRL variants are constructed to match MG token count exactly. NL instructions are allowed to differ in token count from MG, as this difference represents the compression achieved by MetaGlyph.

\paragraph{Reporting.} We report instruction token counts for each condition and compression ratios (NL tokens minus MG tokens, divided by NL tokens).

\subsection{Tasks and Datasets}

We evaluate MetaGlyph across four task families designed to probe distinct instruction-semantic functions. All tasks use automatic evaluation against deterministic gold-standard outputs.

\paragraph{Selection and classification (membership and negation).} Inputs are categorical lists (10--15 items) with ground-truth labels. The instruction requires selecting items that satisfy inclusion and exclusion constraints ($\in$, $\neg$) simultaneously. Gold outputs are computed deterministically from item labels. Metrics: exact match for selected sets.

\paragraph{Structured extraction (constraint plus schema).} Inputs are synthetic technical assessment reports containing risk descriptions with varying properties. The instruction specifies which risk categories to extract and a strict output schema (JSON array with specific fields). Metrics: field-level accuracy, JSON validity.

\paragraph{Constraint composition (conjunction and disjunction).} Inputs are mixed-category lists where multiple constraints apply simultaneously. The instruction tests $\cap$ for conjunctive constraints. Metrics: deterministic constraint satisfaction.

\paragraph{Conditional transformation (implication and composition).} Inputs are account or entity records with categorical properties. The instruction specifies conditional rules using $\Rightarrow$ and composition using $\circ$ for sequencing (for example, ``if employee then label=internal; then normalize names to lowercase''). Metrics: rule-consistency accuracy.

For each family we construct 150 instances with gold outputs and explicit constraint checks.

\subsection{Models and Decoding}

We evaluate across three instruction-tuned language models accessed via OpenRouter API, as shown in Table~\ref{tab:models}.

\begin{table}[htbp]
\centering
\caption{Evaluated Models}
\label{tab:models}
\begin{tabular}{@{}lll@{}}
\toprule
Model & Parameters & OpenRouter ID \\
\midrule
Llama 3.2 3B Instruct & 3B & \texttt{meta-llama/llama-3.2-3b-instruct} \\
Gemma 3 12B IT & 12B & \texttt{google/gemma-3-12b-it} \\
Qwen 2.5 7B Instruct & 7B & \texttt{qwen/qwen-2.5-7b-instruct} \\
\bottomrule
\end{tabular}
\end{table}

All models are run in a single-turn setting with deterministic decoding: temperature 0, top-p 1.0, no frequency or presence penalties, and maximum output length of 2048 tokens.

\subsection{Metrics}

We report three metric classes.

\paragraph{Task performance.} Accuracy, exact match, and F1 depending on task family. For all tasks we additionally report JSON parse success rate.

\paragraph{Semantic equivalence.} The rate at which model outputs under MG instructions match outputs under NL instructions for the same input. This metric captures whether symbolic compression preserves instruction semantics from the model's perspective.

\paragraph{Operator fidelity.} For each operator used in a task we define a concrete verification check:
\begin{itemize}
    \item $\in$ (membership): all selected items belong to the specified category
    \item $\neg$ (negation): no excluded items appear in the output
    \item $\cap$ (intersection): outputs satisfy all conjunctive constraints simultaneously
    \item $\Rightarrow$ (implication): conditional rules are applied to qualifying items
\end{itemize}

\paragraph{Compression descriptors.} We report instruction token count and compression ratio (NL tokens minus MG tokens, divided by NL tokens). These are treated as descriptive measures of semantic density.

\subsection{Reproducibility}

We release prompt templates, the full set of instruction variants, raw model outputs, and evaluation results. The pipeline is implemented as a six-stage process:
\begin{enumerate}
    \item Dataset generation (task instances with gold outputs)
    \item Prompt construction (NL, MG, CTRL variants)
    \item Token accounting
    \item Model execution via OpenRouter API
    \item Automatic evaluation against gold standards
    \item Aggregation and reporting
\end{enumerate}

All artifacts are saved to enable independent verification.

\subsection{Safety and Scope}

All tasks are benign and exclude harmful content. We do not implement or evaluate adversarial prompt strategies, and we do not treat safety bypassing as an objective. Our claims are limited to behavioral equivalence and instruction-level semantic compression under controlled prompting conditions.

\section{Experiments and Results}
\label{sec:results}

\subsection{Token Compression Results}

MetaGlyph achieved substantial and consistent token reduction compared to natural language baselines across all task families, as shown in Table~\ref{tab:compression}.

\begin{table}[htbp]
\centering
\caption{Token Compression by Task Family}
\label{tab:compression}
\begin{tabular}{@{}lcccc@{}}
\toprule
Task Family & NL Tokens & MG Tokens & CTRL Tokens & Reduction \\
\midrule
Selection \& Classification & 215 & 41 & 41 & \textbf{80.9\%} \\
Structured Extraction & 176 & 52 & 52 & \textbf{70.5\%} \\
Constraint Composition & 134 & 48 & 48 & \textbf{64.2\%} \\
Conditional Transformation & 164 & 62 & 62 & \textbf{62.2\%} \\
\bottomrule
\end{tabular}
\end{table}

\textbf{Mean token reduction: 69.5\%}

The compression ratio varies systematically with task complexity. Selection tasks achieve the highest compression (80.9\%) because membership predicates like ``belongs to the category of mammals'' reduce to single symbolic expressions like \texttt{$\in$(mammal)}. Conditional transformation tasks show more modest compression (62.2\%) because rule structure inherently requires more specification even in symbolic form.

\subsection{Semantic Equivalence Results}

We measured semantic equivalence as the rate at which model outputs under MG instructions matched outputs under NL instructions for the same input, \emph{independent of gold-standard correctness}. This metric isolates whether symbolic compression preserves instruction semantics from whether the model performs the underlying task correctly.

Output equivalence (NL $=$ MG) by model and task family is shown in Table~\ref{tab:equivalence}.

\begin{table}[htbp]
\centering
\caption{Output Equivalence (NL $=$ MG) by Model and Task Family}
\label{tab:equivalence}
\begin{tabular}{@{}lccccc@{}}
\toprule
Model & Selection & Extraction & Constraint & Transformation & NL $=$ CTRL \\
\midrule
Llama 3.2 3B & 0\% & 2\% & 0\% & 34\% & \textbf{0\%} \\
Gemma 3 12B & 0\% & 20\% & 0\% & 0\% & \textbf{0\%} \\
Qwen 2.5 7B & 8\% & 44\% & 0.7\% & 0\% & \textbf{0\%} \\
\bottomrule
\end{tabular}
\end{table}

\paragraph{Key findings:}

\begin{enumerate}
    \item \textbf{CTRL fix verified across all models}: After correcting the control prompt construction (replacing ALL operators with nonsense symbols), NL$=$CTRL rates dropped to 0\% across all models and task families. This confirms that the original high NL$=$CTRL rates were artifacts of incomplete symbol replacement, not evidence that models ignore symbolic structure.

    \item \textbf{Semantic preservation varies by task family}: Structured extraction shows the highest NL$=$MG equivalence (Qwen: 44\%, Gemma: 20\%), suggesting membership-based selection tasks are more amenable to symbolic compression. Constraint composition shows near-zero equivalence across all models, indicating compositional operators are poorly internalized.

    \item \textbf{Qwen 2.5 7B achieves highest extraction equivalence (44\%)}: Nearly half the time, MG and NL prompts produce identical outputs for extraction tasks. This demonstrates that MetaGlyph can preserve instruction semantics for mid-size models when operators align with their pretraining exposure.

    \item \textbf{Llama 3.2 3B shows unexpected strength on transformation (34\%)}: Despite being the smallest model, Llama shows substantial semantic preservation on conditional transformation tasks, possibly due to pattern-matching against similar training examples.
\end{enumerate}

\paragraph{Distinguishing semantic preservation from task accuracy:}
The original evaluation conflated two independent questions: (1) does MG preserve NL instruction semantics? and (2) do both conditions produce correct outputs? The 50.5\% equivalence rate for Llama demonstrates that semantic preservation can succeed even when task accuracy is low. Both NL and MG may be ``wrong together'' in the same way, which validates the compression while identifying task difficulty as a separate concern.

\subsection{Parse Success Rates}

\textbf{Gemma 3 12B:} Achieved 100\% parse success across all conditions and task families.

\textbf{Llama 3.2 3B:} Showed variable parse success, as detailed in Table~\ref{tab:parse}.

\begin{table}[htbp]
\centering
\caption{Parse Success Rates for Llama 3.2 3B}
\label{tab:parse}
\begin{tabular}{@{}lccc@{}}
\toprule
Task Family & NL Parse & MG Parse & CTRL Parse \\
\midrule
Selection \& Classification & 32\% & 34\% & 30\% \\
Structured Extraction & 92\% & 92\% & 94\% \\
Constraint Composition & 100\% & 30\% & 28\% \\
Conditional Transformation & 100\% & 100\% & 100\% \\
\bottomrule
\end{tabular}
\end{table}

The parse success disparity reveals an important interaction between model scale and instruction format. Smaller models frequently produced verbose explanatory outputs instead of the requested JSON, especially under symbolic conditions.

\subsection{Operator Fidelity}

Operator fidelity by model is shown in Table~\ref{tab:fidelity}.

\begin{table}[htbp]
\centering
\caption{Operator Fidelity by Model}
\label{tab:fidelity}
\begin{tabular}{@{}lccc@{}}
\toprule
Operator & Llama 3.2 3B & Gemma 3 12B & Qwen 2.5 7B \\
\midrule
$\in$ (membership) & 33.3\% & 0.0\% & \textbf{18.9\%} \\
$\to$ (transformation) & 0.0\% & 0.0\% & 0.0\% \\
$\cap$ (intersection) & 0.0\% & 0.0\% & 0.0\% \\
$\Rightarrow$ (implication) & 0.0\% & 0.0\% & 0.0\% \\
\bottomrule
\end{tabular}
\end{table}

The membership operator ($\in$) shows the highest fidelity across models, with Llama 3.2 3B at 33.3\% and Qwen 2.5 7B at 18.9\%. This is consistent with $\in$'s high frequency in pretraining data across mathematical, programming, and set-theoretic contexts. Complex operators like implication ($\Rightarrow$) and intersection ($\cap$) show zero fidelity across all models, indicating that compositional operator semantics are not robustly internalized from pretraining alone.

Notably, Gemma 3 12B shows 0\% fidelity on all operators despite being the largest model, suggesting that model scale does not automatically improve symbolic operator interpretation---and may even decrease it due to stronger natural language biases from instruction tuning.

\subsection{Response Time Observations}

Preliminary timing data across models suggests that MG prompts tend to yield faster response times due to reduced token counts. Qwen 2.5 7B showed consistent response patterns, as shown in Table~\ref{tab:timing}.

\begin{table}[htbp]
\centering
\caption{Response Time Observations for Qwen 2.5 7B}
\label{tab:timing}
\begin{tabular}{@{}lll@{}}
\toprule
Condition & Mean Response Time & Pattern \\
\midrule
NL & $\sim$3,500 ms & Longer due to verbose prompts \\
MG & $\sim$2,500 ms & $\sim$30\% faster than NL \\
CTRL & $\sim$4,000 ms & Longest; model confusion adds overhead \\
\bottomrule
\end{tabular}
\end{table}

The response time reduction for MG prompts is consistent with shorter prompts reducing processing overhead, though the CTRL condition shows that semantically incoherent symbolic prompts can actually increase processing time, possibly due to the model's attempts to resolve inconsistent instructions.

\section{Analysis}
\label{sec:analysis}

\subsection{Syntactic Scope and Operator Binding}

The low operator fidelity scores reveal that models do not reliably parse symbolic operators as scope-delimited constraints. Consider the selection task operator sequence \texttt{$\in$(mammal) $\cap$ $\in$(pet) $\cap$ $\neg$(bird)}. The intended interpretation is strictly conjunctive: select items that are mammals AND pets AND not birds. However, models often treated this as a disjunctive list or ignored the negation constraint entirely.

This scope-binding failure may arise from how these symbols are encountered in pretraining data. In mathematical texts, $\cap$ typically appears in set-theoretic contexts where both operands are explicit sets. In MetaGlyph, $\cap$ connects constraint predicates rather than sets, requiring generalization from set intersection to predicate conjunction.

\subsection{Compositionality Limits}

The conditional transformation task reveals limits to compositional interpretation. Instructions like:

\begin{lstlisting}
( in(employee) => label=internal ) cap ( in(contractor) => label=external ) circ normalize(name=lowercase)
\end{lstlisting}

require the model to recognize parallel conditional rules, apply them independently, and compose results. The zero fidelity on $\Rightarrow$ indicates that models did not reliably parse implications as conditional rules.

The 34\% semantic equivalence on conditional transformation for Llama 3.2 3B suggests some models occasionally produce correct outputs through alternative reasoning paths---possibly by pattern-matching against similar training examples.

\subsection{Model Scale Effects}

Gemma 3 12B achieved 100\% parse success while Llama 3.2 3B showed significant parse failures in symbolic conditions (30\% for MG in constraint composition). Larger models demonstrate robust ability to infer output format from compact symbolic prompts.

However, larger scale did not improve operator fidelity: Gemma 3 12B showed 0\% fidelity vs.\ Llama 3.2 3B's 33.3\% on membership. This counterintuitive result may indicate that larger models have stronger natural language biases from instruction tuning, treating symbolic operators as formatting rather than semantic primitives.

\subsection{Control Condition Design and Correction}

Initial experiments revealed unexpectedly high NL$=$CTRL equivalence rates, which led to discovery of a critical bug in the control prompt construction.

\paragraph{The Bug:} Control prompts were designed to replace meaningful operators with nonsense symbols ($\odot$, $\otimes$, $\ominus$), but the implementation only replaced SOME operators while leaving others intact:

\begin{lstlisting}
MG:   items -> in(mammal) cap in(pet) cap neg(bird) => select circ sort(name)
CTRL: items -> odot(mammal) otimes in(pet) odot neg(bird) => select circ sort(name)
                           ^valid   ^valid  ^valid ^valid
\end{lstlisting}

This allowed models to parse most of the instruction correctly, invalidating the control condition.

\paragraph{The Fix:} All operators are now replaced with nonsense symbols:

\begin{lstlisting}
CTRL: items -> odot(mammal) otimes odot(pet) otimes ominus(bird) circledast select oplus sort(name)
               ^nonsense throughout
\end{lstlisting}

\paragraph{Implications and Verification:} The high NL$=$CTRL rates in preliminary results (47--100\%) were artifacts of incomplete symbol replacement, not evidence that models ignore symbolic structure. After implementing the fix and re-running experiments with Qwen 2.5 7B (150 instances per family), \textbf{CTRL semantic equivalence dropped to 0\% across all task families}, confirming that properly constructed controls produce substantially different outputs from both NL and MG conditions. This validates the experimental design and isolates the semantic effects of MetaGlyph operators.

\subsection{Operator-Specific Semantic Stability}

Pragmatic stability varies across the operator inventory. Membership ($\in$) shows partial stability, likely because its meaning is reinforced across mathematical, programming, and set-theoretic contexts. Implication ($\Rightarrow$) shows zero stability, possibly due to contextual diversity in pretraining data (mathematical proofs vs.\ programming vs.\ logical specifications).

\section{Discussion}
\label{sec:discussion}

\subsection{Implications for Prompt Design}

The results support a nuanced view of symbolic instruction languages:

\paragraph{Compression benefits are reliable.} MetaGlyph achieves 3:1 or better token reduction across task types.

\paragraph{Parse stability requires model scale.} Smaller models (3B parameters) show significant parse failures with symbolic prompts.

\paragraph{Semantic fidelity is operator-dependent.} Membership ($\in$) shows partial reliability; implication ($\Rightarrow$) and intersection ($\cap$) show zero fidelity.

\paragraph{Hybrid approaches may be optimal.} Symbolic operators for structure combined with natural language anchors for critical constraints.

\subsection{What the Experiments Establish}

\begin{itemize}
    \item MetaGlyph achieves 62--81\% token reduction across four task families
    \item Semantic equivalence (NL $=$ MG) reaches up to 44\% for Qwen 2.5 7B on extraction tasks and 34\% for Llama 3.2 3B on transformation tasks, demonstrating that symbolic compression can preserve instruction semantics independent of task accuracy
    \item CTRL semantic equivalence is 0\% across all models after fixing control prompt construction, validating the experimental design
    \item Model-specific responses to symbolic forms vary by task family, with extraction tasks showing highest semantic preservation across models
    \item Operator fidelity varies by symbol, with membership ($\in$) showing highest reliability (33.3\% in Llama, 18.9\% in Qwen)
\end{itemize}

\subsection{What the Experiments Do Not Establish}

\begin{itemize}
    \item System-level latency or memory improvements (timing data is preliminary)
    \item Generalization beyond tested models
    \item Whether fine-tuning or system prompts could improve operator fidelity
    \item Optimal operator inventories or grammar designs
\end{itemize}

\subsection{Limitations and Stress Testing}

The experiments reported in this paper are intentionally designed to establish a baseline claim: symbolic instruction metalanguages can semantically compress instructions under controlled, long-context conditions. However, they do not exhaustively characterize the limits of such metalanguages. In particular, this work does not include systematic stress testing, and any interpretation of robustness beyond the evaluated tasks would be unwarranted. Stress testing is therefore a necessary next step for understanding where MetaGlyph-style instruction languages break down and why.

Stress testing in this context refers to deliberately constructed cases that isolate known failure modes of instruction following, rather than to adversarial or safety-oriented probing. Several concrete stress-test categories are immediately apparent. First, scope and binding tests can examine whether operators such as $\neg$, $\cap$, and $|$ attach to the intended clause. For example, prompts that distinguish between ``$(A \cup B) \cap \neg C$'' versus ``$A \cup (B \cap \neg C)$'' allow deterministic detection of misbinding through set-based evaluation. Second, nested conjunction--disjunction tests can probe whether exceptions are applied locally or globally, such as ``include $A$, and include $B$ unless $B$ is also $D$.'' Third, quantifier stress tests using $\forall$ and $\exists$ can distinguish whether models correctly separate ``all elements satisfy'' from ``at least one element satisfies,'' which can be verified automatically using group-level gold labels.

Additional stress tests target operator-specific behavior. Implication direction tests can determine whether $\Rightarrow$ is treated as a one-way rule or incorrectly as an equivalence, by constructing inputs where the contrapositive should not hold. Composition-order tests using $\circ$ can check whether sequential operations are applied in the correct order by embedding inputs where normalization changes extraction outcomes. Symbol polysemy tests can assess whether characters such as $|$ are interpreted semantically or degraded to formatting markers. Finally, operator-density tests can establish practical compression limits by gradually increasing symbolic density until performance collapses, thereby identifying a usable semantic-density ceiling.

These stress tests are not required to validate the central claim of this paper, but they are essential for turning symbolic instruction metalanguages into reliable design tools. By making failure modes explicit rather than implicit, future work can define safe operating regimes, refine operator grammars, and determine which symbols are robust enough to support further semantic compression.

\section{Conclusion}
\label{sec:conclusion}

This work introduced MetaGlyph, a symbolic instruction metalanguage designed to semantically compress natural-language prompts using mathematical and logical operators with high pretraining exposure. The central empirical question was whether such operators function as intuitive instruction-semantic primitives that models can interpret without explicit teaching or system-level decoding schemes.

The experiments provide a nuanced answer that depends critically on separating two questions: (1) does symbolic compression preserve instruction semantics? and (2) do models execute the compressed instructions correctly?

\paragraph{On semantic preservation:} The experiments show task-dependent semantic preservation. Qwen 2.5 7B achieves 44\% output equivalence between MG and NL conditions on extraction tasks, while Llama 3.2 3B achieves 34\% on transformation tasks. This demonstrates that MetaGlyph can function as a lossless compression of instruction semantics for specific operator-task combinations, even when neither condition produces correct outputs. The compression is valid; the model capability on the underlying task is the limiting factor.

\paragraph{On control condition validity:} After correcting the control prompt construction (replacing ALL operators with nonsense symbols), CTRL semantic equivalence dropped to 0\% across all models and task families. This validates the experimental design: models do respond differently to semantically coherent versus incoherent symbolic instructions, confirming that the observed NL$=$MG equivalence reflects genuine semantic preservation rather than superficial pattern matching.

\paragraph{On operator fidelity:} The membership operator ($\in$) shows the highest reliability across models (33.3\% in Llama, 18.9\% in Qwen), consistent with its frequency and semantic stability in pretraining data. Compositional operators like implication ($\Rightarrow$) and intersection ($\cap$) show zero fidelity across all models, indicating these symbols do not function as robust instruction primitives without explicit specification.

These findings carry practical implications for prompt engineering. Symbolic compression is viable for reducing token costs and may improve processing speed, making it attractive for cost-sensitive or latency-sensitive applications. However, practitioners should not assume that compressed prompts preserve fine-grained instruction semantics. Testing and validation are essential, particularly for operators beyond basic membership. Hybrid approaches that combine symbolic structure with natural language specification of critical constraints may offer the best tradeoff between compression efficiency and semantic reliability.

The research identifies productive directions for future investigation. Stress testing should systematically characterize the operating envelope of symbolic instruction languages, probing scope binding limits, compositional depth limits, and operator-density thresholds. Comparative studies should examine whether system prompts defining operator meanings, few-shot examples demonstrating operator usage, or targeted fine-tuning can improve operator fidelity. Extended evaluations should test MetaGlyph across larger frontier models and specialized reasoning models that may have different pretraining exposure to symbolic notation.

Ultimately, the question of whether symbolic operators function as instruction-semantic primitives depends on how models are trained and how symbols are distributed across pretraining data. Symbols that appear frequently with consistent meaning across diverse contexts---like $\in$ for membership---show partial semantic stability. Symbols with more varied contextual usage---like $\Rightarrow$ for implication---do not reliably transfer their formal meaning to instruction contexts. Designing effective symbolic instruction languages will require careful empirical assessment of which symbols carry robust pretrained semantics and which require explicit specification or augmentation.

The MetaGlyph framework demonstrates that significant instruction compression is achievable with current models, even without perfect semantic preservation. As models continue to scale and training data increasingly includes structured symbolic content, the gap between syntactic parsing and semantic interpretation may narrow. This work provides a foundation for tracking that evolution and for developing symbolic instruction languages that fully realize the efficiency potential of compact, compositional instruction specification.

\section*{Code and Data Availability}

The complete experimental pipeline, prompt templates, raw model outputs, and evaluation scripts are publicly available at: \url{https://github.com/BEAR-LLM-AI/arXiv-Semantic-Compression-of-LLM-Instructions-via-Symbolic-Metalanguages}

\bibliography{references}

\end{document}
